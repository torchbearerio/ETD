\chapter{Background} \label{CH:background}

\section{Distraction and Cognitive Load In the Context of Driving}\label{Sect:distraction}

While driving is a dangerous endeavour due to a wide array of factors, including environmental, human and vehicle equipment related circumstances, a significant contributor is driver distraction, which accounts for 16 percent of vehicle accidents \cite{distracted_nhtsa_17}. Distraction, in the context of driving, is the diversion of attention away from the task of safely and efficiently operating the vehicle, onto some secondary task \cite{regan2011driver}. If we consider the driving task to consist of applying lateral (right and left steering) and longitudinal (braking and forward acceleration), then distraction is dangerous primarily because it inhibits the driver's ability to quickly and accurately apply these actions in response to changing situations in the environment \cite{pettitt2005defining}.

\subsection{Sources of Distraction}
Broadly, a source of distraction is classified as in-vehicle or out-of-vehicle. Out-of-vehicle distractions include visually abnormal occurrences such as police actions, accidents, or billboards \cite{edquist2011effects}. In-vehicle distractions can be further refined as technology-based or non-technologically based. Talking with a passenger, applying makeup, eating, or smoking all pose a potential non-technological distraction. Technological distractions are receiving rapidly increasing academic attention due to the rising penetration of in-vehicle information systems (IVIS) and smartphones \cite{bayly200812}. IVIS pose a significant issue in regards to distraction, as they often require the driver to look at a screen, or interact with the system in some way, creating both a visual and cognitive distraction \cite{birrell2011impact}. Cognitive distraction results in unsafe driving behavior, including steering errors (lane departures), increased variability in accelerator position, and the sharp breaking due to a shorter window in which to respond to a change in the environment \cite{reyes_influence_2004}. Mobile devices, such as smartphones, lead to driver distraction via the introduction of a physical (holding and tapping/swiping) visual and cognitive load upon the driver One study estimates an increase in reaction time to a pedestrian crossing the path of travel of 204 percent when the driver attempts to text and drive. \cite{CHOUDHARY2017351}.

Navigation systems, implemented via IVIS, or a mobile device, represent a unique form of distraction in that the interaction with the system (supplying a destination, looking at a map, listening to instructions) presents one secondary task, while the execution of the system's instructions (scanning for upcoming turns) presents another. Together these tasks can cause the driver to disengage from the environment \cite{leshed2008car}. This disengagement leads to an increase in reaction time while using a navigation system, which is more pronounced for navigation apps that have a visual interface than those which are entirely audio-based \cite{harms2003peripheral}. The task of entering an address using a touch screen poses a particular problem, with one study finding a increase in the standard deviation of lateral vehicle position of 60 percent. \cite{tsimhoni2004address}.

\section{Landmarks in Navigation}

Mainstream navigation aids tend to heavily utilize distance-to-street-name instructions, which require the driver to conceptualize distances and perform a visual search for small road signs. \cite{burnett2000turn}. Humans, on the other hand, tend to provide navigation instructions using landmarks \cite{zaidel1997automatic}. One study found that instructions provided by a passenger, which were primarily landmark-based, resulted in fewer navigation errors, shorter trip duration, lower perceived workload and a higher quality of driving as rated by an expert, leading to the conclusion that the inclusion of landmarks in automated navigation instructions could be beneficial \cite{burnett1997assessment}. 

Lovelace \cite{lovelace1999elements} examines the components of good navigation instructions for both familiar and unfamiliar routes.  They found that in general more information provided in an instruction resulted in higher perceived quality. Additionally, they found that the inclusion of landmarks, both at maneuver points and intermittently along the route, significantly increased perceived route quality.

Golledge \cite{golledge2003human} asserted that landmarks can aid in the navigation task because they serve as both global reference point, allowing the driver to mentally organize the space he is traveling through, and also as a sort of marker for decisions (maneuver) points. Indeed, the substitution of distance-based instructions for landmark-based instructions has been shown to decrease navigation error count as well as driver confidence \cite{may2005incorporating}. Interestingly, while the quality of landmarks did have a significant effect on these measures, both good and poor landmarks were significantly better than distances alone \cite{may2005incorporating}. Completing a study in a real traffic environment, another work found that the use of landmarks (as opposed to distance) resulted in fewer glances at the navigation aid's display and better driving performance as measured by lane departure count and improper turn signal use. \cite{may2006presence}.

\section{Landmark Saliency: What Makes A Good Landmark}\label{Sect:eqns}
\textit{Saliency} is to the property of being particularly noticeable, prominent or important \cite{saliency_dict}. A landmark is a physical feature that serves as a point of reference within the environment; it is distinctive from its surroundings to such a degree that it is easily recognizable and represents an exact point in space. Because of this importance of uniqueness, the saliency of a landmark is not a function of the attributes of an individual landmark but rather how distinctive those features are relative to nearby objects. Indeed, being a good, salient landmark is a \textit{relative} property \cite{raubal2002enriching}. 

Landmarks can be broadly classified as \textit{global}, visible form the entire route and relevant throughout, or \textit{local}, important to a specific maneuver point (turn). Driving directions do not usually include global landmarks \cite{sorrows1999nature}. Local landmarks are best for navigation, and are most useful to the driver near decision (maneuver) points \cite{lee2003effect}.

Saliency is represented by a tripartite typology, where three distinct dimensions, \textit{visual}, \textit{semantic} and \textit{}{structural}, compose the overall saliency of a landmark \cite{sorrows1999nature}. 

\subsection{Visual Saliency}
 Visually saliency is analogous with visual attractiveness. In general, visual saliency is based on behavior observed in most vertebrates, in which they alter their gaze so as to focus more attention on relevant details in a scene while ignoring unimportant areas \cite{harel2007graph}. A region within in the scene, or a specific object in the space, is salient if it receives a significant portion of attention. In the context of navigation, a landmark is visually salient if it has sharp contrast with its surroundings and is prominent (easily in view) from the driver's location \cite{sorrows1999nature}. 

Reubel and Winter \cite{raubal2002enriching} show that the visual saliency of a landmark is calculated by comparing several physical properties. (Of course, the calculated value for a landmark has no meaning until compared to that of a nearby landmark--saliency is relative.) The \textit{facade area} represents the total physical area that is visible to the driver. (Essentially, the bigger the landmark, the better.) The oddity of the shape also plays a role; the larger the deviation between the shape of the landmark's silhouette and a rectangle, the more visually attractive it is. Color is the final factor, specifically how different the landmark's color is from the surroundings.

\subsection{Semantic Saliency}

Sorrows and Hirtle \cite{sorrows1999nature} define what they coin a \textit{cognitive landmark}, a landmark whose meaning, history or cultural importance makes it prominent in the environment. Such a landmark has an atypical level of importance relative to its surroundings, possibly in spite of a typical level of visual attraction. The house of a university president, for example, likely has a high degree of semantic saliency due to its significance in the community, even if visually it may be quite similar to surrounding homes.

Reubel and Winter \cite{sorrows1999nature} refine the notion of a cognitive landmark to obtain a more formalized definition of semantic saliency. Specifically, they include a Boolean value for whether, or not, a landmark has historical or cultural significance to the area. Additionally, they include a Boolean value for whether the landmark his discernible commercial semantics, that is, is it a business of a type people are familiar with (such as coffee houses or grocery stores.)

Duckham and Winter \cite{quesnot2012linked} expand this definition by suggesting that the semantic saliency of a landmark is also a function of its ubiquity. The ubiquity of a landmark is important, they argue, cultural significance is less meaningful to people unfamiliar with a given area, as what is culturally significant to the area may be unknown to them. Accounting for ubiquity in the semantic saliency measure accounts for the fact that the more instances of a landmark there are, the more widespread its significance is. As an example, consider a 50-year old local burger joint situated near a McDonald's that opened a year ago: while the cultural and historical significance is much higher for the burger joint at the local level, the ubiquity of the McDonald's belies its much higher significance on the global level.

Geosocial data streams, such as FourSquare, Facebook and Google Places also have the potential to provide semantic saliency information. Quesnot and Roche \cite{quesnot2014measure} argue that geosocial data, which encodes information about who visits a landmark, can offer valuable insight into the importance of that landmark. If a large number of people frequently visit a landmark, it is likely to be more important than one which receives few visitors. It essentially acts as a proxy for cultural significance, with the enhancement that it provides a quantitative, real-time measure. 

Uniqueness is also an important component of semantic saliency \cite{caduff2008assessment}. Just as a green house is visually salient among a group of red houses, a library is semantically salient among a group of restaurants. The uniqueness of a landmark's intended purpose within its surroundings is an important consideration \cite{caduff2008assessment}.

\subsection{Structural Saliency}
The final tenant of landmark saliency is structural saliency, which broadly refers to the pertinence of a landmark in the context of its location in the physical space of its surroundings \cite{sorrows1999nature}. At a more applied level, a landmark is structurally salient if its location (relative to the route) is easy to conceptualize cognitively and linguistically \cite{klippel2005structural}. 

Klippel and Winter \cite{klippel2005structural} developed the first formal syntax for structural saliency. They provide a hierarchy of structural saliency in terms of a landmark's position in relation to the intersection where a turn is to occur. While the hierarchy is extremely thorough, the key takeaway is that it is best for a landmark to be located on the corner of an intersection where a turn is to occur. The location of such a landmark is easy to describe linguistically: ``turn left after the McDonald's" or ``turn left before the McDonald's", depending on whether the landmark is on the near or far side of the intersection. If a landmark is located significantly before, or after, the entire intersection, then it becomes difficult to summarize into an instruction, and potentially even more difficult for a driver to conceptualize. Instructions such as``at the intersection after where the McDonald's is" are more complex both linguistically and conceptually. Roser \cite{roser2012structural} offers empirical evidence, based on an ergonomic study in a virtual environment, which supports this hierarchy.

\section{Prior Art: Automated Landmark Detection}\label{Sect:eqns}
Multiple approaches have been implemented in attempts to automatically select landmarks for navigation, spanning a wide range of goals, working definitions of landmark saliency and data sources. Much work has also been done in the context of pedestrian navigation, to a greater extent than has been done for vehicle-based navigation.

Hile et al \cite{hile2008landmark} leverage a dataset of geotagged images to generate landmarks for pedestrian walking instructions. For a given path a pedestrian will walk, a database of points of interest is used to select and annotate an image. The photograph, along with the description and navigation instruction, are displayed on the user's device. Selection criteria is based on the proximity of a landmark to the user's path of travel, as well as how closely the angle of the photograph matches the heading the user is traveling.

Beharee and Steed \cite{beeharee2006natural} also used geotagged images to provide navigation aid to pedestrians, but selected a series of landmark photos to show along each leg of the route. Proximity to the route was used as the selection criteria. Landmarks were not given lexical descriptions. A between-subjects study revealed that in areas not familiar to the subject, the addition of photographs to the navigation application allowed subjects to arrive at target destinations in less time than when with textual directions alone.

In another application targeted at pedestrians, Wenig et al \cite{wenig2017pharos} developed a system for finding global landmarks that can be used to orient the user. For example, a user looking for a destination in Paris might be given instructions in terms of the relative location of the Eiffel Tower. Global landmarks are used based on the authors' argument that local landmarks are difficult to select accurately. Candidate landmarks for a given region are predefined; the best landmark is chosen based on level of visibility throughout the entire route to be traveled. The visibility of a landmark at a given point is determined in a binary fashion using Google Street View images and a deep neural network. The authors show that this approach leads to greater confidence and more accurate cognitive map building among subjects.

Elias and Brenner \cite{elias2005automatic} use visual saliency to select landmarks for driving-based navigation instructions. Using a Geographic Information System (GIS) dataset, the authors mine candidate landmarks (always buildings), where a landmark is a candidate if it has some unique or distinctive feature compared to its surroundings. Features examined include building use or purpose, land use type and building extremities, such as outbuildings or carports. The best landmark is chosen based on how visible it is to the driver as she approaches; this is determined using a three-dimensional aerial laser scanning model of the area and modeling the the area of a landmark which is within the drivers cone of sight. The system does not offer detailed landmark descriptions, and was not evaluated by a human-based experiment. Torchbearer provides meaningful landmark descriptions via human and algorithmic input, and we perform a small-scale but thorough field study with human subjects.

\section{Electronic Navigation Aids}\label{Sect:eqns}

There currently exist a number of commercial, as well as academic or open-source, electronic navigation platforms. Most provide only distance-based instructions, but some prototypes do exist which incorporate some form of landmark descriptions (especially among systems designed for pedestrian use.)

\subsection{Google Maps}
Google Maps is a mobile application for iOS and Android devices which is capable of providing turn-by-turn driving directions between an origin and destination point. Users provide the destination via voice or keyboard, and can enter addresses, coordinates or points-of-interest. The app provides primarily distance-based instructions, complete with street names. Some instructions will use road topology to describe the maneuver point, such as "turn left at the end of the road."

Routing is based on finding the shortest travel time, and includes traffic and construction delays in its optimization.

As of mid-2018, Google Maps has, reportedly, begun to include landmarks into its spoken directions \cite{fingas_2018}. It is not yet a documented feature, and has been enabled on only a small number of devices. It remains unclear what types of landmarks it incorporates and what methods it uses for selection \cite{fingas_2018}.

\subsection{Waze}
Waze, owned by Google since 2013 \cite{schneider_2014}, provides turn-by-turn navigation instructions in a similar manner to Google. Waze is novel because, along with a base of OpenStreetMap data, Waze considers travel time, police traps, construction delays and other data from its users, which it incorporates into its map and routing decisions. Spoken instructions consist of distances and street names.