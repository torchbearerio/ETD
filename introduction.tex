\chapter{Introduction}\label{CH:introduction}
In 2016, there were nearly 35,000 deaths resulting from motor vehicle accidents \cite{crashes_48}. Yet despite the danger of driving, automobile transportation remains an integral part of people's’ daily lives: in that same year, Americans drove a collective 3.17 trillion miles \cite{crashes_48}.

A large majority of automobile fatalities are the consequence of driving under the influence, adverse weather conditions or speeding. However, in 2016, 16 percent of all vehicle accidents were the result of driver distraction \cite{distracted_nhtsa_17}.  Tasks which a driver must perform in conjunction with operating a vehicle (secondary tasks) impose cognitive load, which in turn leads to the driver being distracted from vehicle operation. Distraction leads to dangerous driving behavior, such as hard braking, manifested as sharp changes in longitudinal acceleration, or sudden steering corrections, resulting in sharp lateral acceleration \cite{harbluk_distraction_2002}. 

Some secondary tasks, such as texting or applying makeup, are best refrained from altogether. However, other secondary tasks are requisite to the primary task of driving from origin to destination. The use of electronic, turn-by-turn navigation aids, such as Google Maps, is one such task: while it has been shown to produce a significant cognitive load \cite{young_regan_distraction_2007}, it is a valuable tool which allows drivers to efficiently reach a destination. Indeed, in-car navigation is a common task; 67-percent of smart phone users indicate that they use their device for this purpose \cite{pew_smartphone}. Be it utilizing an alternate route to work to avoid construction, trying to find the newest restaurant, or getting from the airport to a hotel in a never-before-visited city, the real-time auditory directions offered by navigation aids have done away with the complexity and stress that comes from needing to memorize a route from a paper map prior to the trip \cite{walker1991vehicle}. By reducing the cognitive load induced by navigation aids, drivers will be enabled to exhibit safer vehicle operation characteristics while still enjoying the benefits of turn-by-turn navigation.

Instructions delivered by the most popular navigation aids generally consist of street names and numeric distances, requiring the driver to perform a visual search for small street name signs and estimate distances. The addition of landmark descriptions could lessen this cognitive load, for example “turn right at the Dairy Queen” instead of “turn right in 600 feet”. A salient landmark, here “Dairy Queen”, provides more obvious information than the numeric distance. Even if a person is driving in a city previously unknown to them, the distinctive appearance of a Dairy Queen transcends geographic location. Previous research has suggested that if electronic navigation aids could include relevant landmarks in their instructions, the cognitive load of the driver could be decreased \cite{burnett2000turn}.

Including landmarks in navigation instructions requires several computational frameworks: first, a method for locating candidate landmarks--physical features located near a maneuver point. Second, a means to lexically describe a landmark, in a detailed manner which allows the driver to easily visualize and locate it. Lastly, a computational approach is needed for determining the best landmark out of a set of candidates--the landmark which is most recognizable to the driver. 

Current approaches to automated landmark-based navigation are limited, many being restricted to pedestrian scenarios, others relying on pre-compiled sets of landmarks and still others using only point-of-interest datasets for selection, without incorporating visual analysis of maneuver points. We present Torchbearer, a system which leverages multiple approaches, or “pipelines”, to locate candidate landmarks, provide lexical descriptions of the same and determine which landmark is best-suited to be included as part of a verbal navigation instruction delivered to a driver. Given the coordinates of an origin and destination, Torchbearer leverages standard pathfinding algorithms to find the least-cost (fastest) route. For each point where the end user will need to perform a driving maneuver, such as a turn or merge, Torchbearer determines the landmark best suited for helping the end user locate that point. It then builds a verbal instruction, inclusive of the street name, distance, description of maneuver to be executed, and description of the landmark, delivered to the driver via a audio-based mobile application. The system extends existing navigation technology to offer landmark-based navigation assistance.
 
A core tenant of Torchbearer's novelty comes form its hybrid, pipeline-based approach: the process of finding landmarks and selecting the most suitable for a given maneuver point is accomplished via four different pipelines: first, a fully human-based approach, which uses crowdsourcing to to find landmarks near a location, select that which is best suited for navigation and generate a description of the landmark. Second, a human in the loop approach, which uses a state-of-the-art saliency detection algorithm to find the most obvious, easiest-to-see landmark, but leverages crowdsourcing to generate a description of that landmark. Third, a pipeline that uses a database of local businesses and points of interest, as well as a deep learning-based object detection algorithm, to find landmarks, and utilizes crowdsourcing to select the optimal one. And lastly, a fully-automated pipeline which uses the saliency-detection algorithm for finding the most visible, easiest to spot landmark and the point-of-interest data source and object detection algorithm to describe that landmark.

Torchbearer differs from existing solutions in three principal aspects: first, its pipeline-based approach allows for several landmark selection methodologies to be used and analyzed interchangeably. Second, it incorporates multiple landmark features into its selection process--visual, data-based and human recognition; this allows Torchbearer to consider a wider range of landmark types than previous systems. Additionally, Torchbearer relies only on publicly available data sources which have very wide geographic coverage across the United States.

We evaluate the system in situ using a standard Peripheral Detection Task (PDT) to measure cognitive load and the NASA Task Load Index survey to measure perceived workload. Additionally, we monitor extreme acceleration occurrences, as an indicator of driving behavior associated with distraction. We also survey subjects on their perception of landmark goodness and ease of navigation. We find............

In addition to a naturalistic human factors study, we also provide a comparative analysis of pipeline performance, examining cost, runtime and result similarity. We find..........

