% Abstract 
The task of navigation adds cognitive distraction to the already demanding task of driving. Most popular navigation aids provide verbal directions based solely on distances and street names, but the inclusion of landmark descriptions in these instructions can improve navigation performance, decrease unsafe driving behaviors and reduce cognitive load. Current approaches to selecting landmarks and building landmark-based instructions rely on a single source of data, thereby limiting the set of potential landmarks, or use a single factor in choosing the best landmark, failing to account for all characteristics that make a landmark suitable for navigation. We develop a multi-pipeline system that leverages both human (crowd-sourced) input and machine-based approaches to find, describe and choose the best landmark. Additionally, we develop a mobile application for the delivery of navigation instructions based on landmarks. We evaluate the cost and performance differences between these pipelines, as well as study the effect of landmark navigation prompts on cognitive load, safe driving behavior and navigation performance via an in situ experiment.
